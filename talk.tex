% (c) 2010 Konstantin Sering <konstantin.sering <aet> gmail.com>
% (cc)-by -- Licenced under Creative Commons Attribution unported
% (http://creativecommons.org/licenses/by/3.0/)
%
\documentclass[xcolor={fixpdftex,hyperref,x11names},10pt,pdftex,hyperref={pdftex}]{beamer}
 
% Pakete
\usepackage[utf8]{inputenc}
\usepackage{lmodern}
\usepackage[T1]{fontenc}
\usepackage[english]{babel}
\usepackage{graphicx}
\usepackage{booktabs}
\usepackage{amssymb}
\usepackage{amsmath}
\usepackage{tikz}
\usepackage{fancyvrb}
\usepackage{color}
\usepackage{listings}
%\usepackage{apacite}
\usepackage{amsmath}
%\usepackage{txfonts}
%\usepackage{showframe}
%\useoutertheme{tree}
\useinnertheme{circles}
\usecolortheme{orchid}
\usecolortheme{whale}
%\usecolortheme{blue}
\usefonttheme[onlymath]{serif}
\usetikzlibrary{arrows,shapes}

% Informationen
\title{Applying IICM to simulated 3D IDM}
\subtitle{ESMP 2013 -- Group 7}
\author{Micha, Michael, Marilisa, Jan, and Tino}
\institute{European Union -- Oulanka Research Station}
\date{23.03.2013}

\setbeamertemplate{navigation symbols}{} 
\setbeamertemplate{footline}{\hspace*{.5cm}\small{\hspace*{50pt} 
\hfill\insertframenumber/\inserttotalframenumber\hspace*{.5cm}}} 
%\setbeamertemplate{headline}{} % removes the headline that infolines inserts

% Sonstiges
\renewcommand{\emph}[1]{\color{red}#1 \color{black}}

\lstset{language=python,%
  basicstyle=\ttfamily\small\color{blue!50!black},%
%  frame=single,%
  commentstyle=\slshape\color{gray!40!black},%
  keywordstyle=\bfseries\color{red!50!black},%
%  identifierstyle=\color{blue!50!black},%
%  stringstyle=\color{red!90!black},%
%  numbers=left,numberstyle=\tiny,%
  basewidth={.5em, .4em},%
  showstringspaces=false,%
  emphstyle=\color{red}}

\begin{document}

\maketitle

\section{Motivation}
\label{sec:motivation}



\begin{frame}
  \frametitle{Goal}
  \begin{columns}
  \begin{column}{0.30\textwidth}
  \includegraphics[width=0.7\textwidth]{figs/checkbox-unchecked.png}
  \end{column}
  \begin{column}{0.70\textwidth}
  With an extended IDM model, simulate human behaviour in a three-alternative temporal judgment task
  \end{column}
  \end{columns}
  \begin{columns}
  \begin{column}{0.30\textwidth}
  \includegraphics[width=0.7\textwidth]{figs/checkbox-unchecked.png}
  \end{column}
  \begin{column}{0.70\textwidth}
  Fit Independent-Channels Model (Miguel \& Rocio) to our simulated data 
  \end{column}
  \end{columns}
\end{frame} 




%%%%%%%%%%%%%%%%%%%%%%%%%%%
\section{Temporal Judgment Tasks}
\label{sec:tjt}
%\section{Decision Tasks}

\begin{frame}
    \frametitle{General Idea}

    \begin{itemize}
        \item Observer's ability to elucidate whether two sensory stimuli
            occurred simultaneously or not.
        \item measured as a function of stimulus onset asynchrony (SOA)
        \item[$\rightarrow$] here audiovisual case (visual regarded as reference)
        \item independent variable is auditory delay

    \end{itemize}

\begin{center}
% nice picture of SOA
    \includegraphics[width=0.7\textwidth]{figs/SOA-in-task.png}
\end{center}

%%%%%%%%%%%%%%%%%%%%%%%%%%%%%
\end{frame}


\begin{frame}
    \frametitle{Synchrony judgment (SJ)}

    \begin{itemize}
        \item Three possible responses (SJ3):
            \begin{enumerate}
                \item \textit{audio-first} (AF)
                \item \textit{video-first} (VF)
                \item \textit{synchronous} (S)
            \end{enumerate}

        \item[$\rightarrow$] Psychometric function: proportion of responses of
         each type as a function of auditory delay.

    \end{itemize}


    \begin{center}
        \includegraphics[width=0.7\textwidth]{figs/SJ3-psychometric-function.png}
    \end{center}
\end{frame}


%%%%%%%%%%%%%%%%%%%%%%%%%%%%%

%\begin{frame}
%    \frametitle{Quantitative Indices}

%    \begin{itemize}
%        \item Two quantitative indices obtain from such tasks
%            \begin{enumerate}
%                \item \textit{Point of subjective simultaneity}: midpoint of the synchrony interval. \textit{AF} and \textit{VF} are equally frequent.
%                \item \textit{synchrony range}: width of interval of \textit{S}-responses  (important for simulation)
%            \end{enumerate}

%    \end{itemize}
%
%%    \begin{center}
%%        \includegraphics[width=0.7\textwidth]{figs/SJ3-psychometric-function.png}
%%    \end{center}

%\end{frame}



%%%%%%%%%%%%%%%%%%%%%%%%%%%%%

\section{Ising Decision Making (IDM)}
\label{sec:idm}

\begin{frame}
    \frametitle{Recap on 2D IDM}

\begin{columns}
        \column{.5\textwidth}
    \begin{itemize}
        \item Two pools of $N_p$ neurons
        \item All thresholds $\Theta$
        \item Internal excitation: $ w_{p}^{+} > 0$
        \item Mutual inhibition: $ - w_{12}^2 < 0$
        \item External exciting fields $ b_p > 0 $
        \item Energy function:
    \end{itemize}
        \column{.5\textwidth}
                \includegraphics[width=0.99\textwidth]{figs/2D-IDM.png}
\end{columns}


\begin{equation}
\begin{split}
E_{IDM}(S) = &-w_1^{+} \sum_{i<j} S_{1i} S_{1j} - w_2^{+} \sum_{i<j} S_{2i} S_{2j} + w_{12}^{-} \sum_{i,j} S_{1i} S_{2j} \\
&-(b_1 - \Theta) \sum_{i} S_{1i} - (b_1 - \Theta) \sum_{i} S_{2i}
\end{split}
\end{equation}

\end{frame}

%%%%%%%%%%%%%%%%%%%%%%%%%%%%%%%
\begin{frame}
    \frametitle{Recap on 2D IDM}

    \begin{itemize}
        \item From Energy Function $E_{IDM}(S)$ to Free Energy Function
            $F_{IDM}(y)$ with $y_p = \frac{\sum\nolimits_{i} S_{pi}}{N_p}$
    \end{itemize}


\begin{equation}
\begin{split}
F_{IDM}(y) = &-W^{+} + y_1^2 - W^+ y_1 y_2 - (B_1 - \Theta) y_1 - (B_2 - \Theta) y_2 \\
&+ \beta^{-1} \frac{N}{2}\sum_{p} (y_p \log (y_p) + (1-y_p) \log(1-y_p))
\end{split}
\end{equation}

\end{frame}


%%%%%%%%%%%%%%%%%%%%%%%%%%%%%

\begin{frame}
    \frametitle{Recap on 2D IDM}

\begin{columns}
        \column{.5\textwidth}
        Clear Stimulus
    \begin{center}
        \includegraphics[width=0.99\textwidth]{figs/clear-stim.png}
    \end{center}

        \column{.5\textwidth}
        Unclear Stimulus
            \begin{center}
            \includegraphics[width=0.99\textwidth]{figs/unclear-stim.png}
        \end{center}

\end{columns}

\end{frame}

%%%%%%%%%%%%%%%%%%%%%%%%%%%%%

\begin{frame}
    \frametitle{Recap on 2D IDM}
    One trial
        \begin{center}
            \includegraphics[width=0.99\textwidth]{figs/surface-trial.png}
        \end{center}
\end{frame}

%%%%%%%%%%%%%%%%%%%%%%%%%%%%

\begin{frame}
    \frametitle{Recap on 2D IDM}
    Traces
        \begin{center}
            \includegraphics[width=0.99\textwidth]{figs/2D-trace.png}
        \end{center}
\end{frame}

%%%%%%%%%%%%%%%%%%%%%%%%


\begin{frame}
  \frametitle{Goal}
  \begin{columns}
  \begin{column}{0.30\textwidth}
  \includegraphics[width=0.7\textwidth]{figs/checkbox-unchecked.png}
  \end{column}
  \begin{column}{0.70\textwidth}
  With an extended IDM model, simulate human behaviour in a three-alternative temporal judgment task
  \end{column}
  \end{columns}
  \begin{columns}
  \begin{column}{0.30\textwidth}
  \includegraphics[width=0.7\textwidth]{figs/checkbox-unchecked.png}
  \end{column}
  \begin{column}{0.70\textwidth}
  Fit Independent-Channels Model (Miguel \& Rocio) to our simulated data
  \end{column}
  \end{columns}
\end{frame}

%%%%%%%%%%%%%%%%%%%%%%%%%%%

% tino

%%%%%%%%%%%%%%%%%%%%%%%%%%%%

\begin{frame}
  \frametitle{3D IDM}
  Extend the Ising decision model to three dimensions.

  \begin{figure}[h]
    \centering
    \includegraphics[width=\textwidth]{./figs/decision_space_3d.pdf}
    \caption{Three dimensional decision space.}
    \label{fig:3dspace}
  \end{figure}
\end{frame}

\begin{frame}
  \frametitle{Video}
  Video :)
\end{frame}

\begin{frame}
  \frametitle{3D IDM Free Energy}
  We have the free energy from the two dimensional Ising decision model:

\begin{equation*}
    \begin{split}
    W^-_{12} y_{1} y_{2} - W^+_{1} y_{1}^{2} - W^+_{2} y_{2}^{2} - y_{1}
    (B_{1} - \Theta_{1}) - y_{2} (B_{2} - \Theta_{2})\\
    + \beta \cdot [N_{1} (y_{1} \operatorname{log}(y_{1}) + (- y_{1} + 1)
    \operatorname{log}(- y_{1} + 1))\\
    + N_{2} (y_{2} \operatorname{log}(y_{2}) + (- y_{2} + 1)
    \operatorname{log}(- y_{2} + 1))]
    \end{split}
\end{equation*}

\begin{equation*}
    \begin{split}
    W^-_{12} y_{1} y_{2} + W^-_{13} y_{1} y_{3} + W^-_{23} y_{2} y_{3} -
    W^+_{1} y_{1}^{2} - W^+_{2} y_{2}^{2} - W^+_{3} y_{3}^{2} \\
    - y_{1} (B_{1} - \Theta_{1}) - y_{2} (B_{2} -
    \Theta_{2}) - y_{3} (B_{3} - \Theta_{3}) \\
    + \beta \cdot [N_{1} (y_{1} \operatorname{log}(y_{1}) + (- y_{1} + 1)
    \operatorname{log}(- y_{1} + 1))\\
    + N_{2} (y_{2} \operatorname{log}(y_{2}) + (- y_{2} + 1)
    \operatorname{log}(- y_{2} + 1))\\
    + N_{3} (y_{3} \operatorname{log}(y_{3}) + (- y_{3} +
    1) \operatorname{log}(- y_{3} + 1))]
    \end{split}
\end{equation*}
\end{frame}

\section{Simulations}
\label{sec:simu}

\begin{frame}[fragile]
  \frametitle{Going Python}
  \begin{lstlisting}
from sympy import symbols, log

Wplus1, Wplus2, Wplus3  = symbols("Wplus1, Wplus2, Wplus3")
Wminus12, Wminus13, Wminus23  = symbols(
                            "Wminus12, Wminus13, Wminus23")
B1, B2, B3 = symbols("B1, B2, B3")
THETA1, THETA2, THETA3 = symbols("THETA1, THETA2, THETA3")
beta = symbols("beta")
y1, y2, y3 = symbols("y1, y2, y3")
N1, N2, N3 = symbols("N1, N2, N3")

F_IDM = (-Wplus1*y1**2 - Wplus2*y2**2 - Wplus3*y3**2
    + Wminus12*y1*y2 + Wminus13*y1*y3 + Wminus23*y2*y3
    - (B1 - THETA1)*y1 - (B2 - THETA2)*y2 - (B3 - THETA3)*y3
    + 1/beta * (N1*(y1*log(y1) + (1-y1)*log(1-y1))
                + N2*(y2*log(y2) + (1-y2)*log(1-y2))
                + N3*(y3*log(y3) + (1-y3)*log(1-y3))))
  \end{lstlisting}
\end{frame}

\begin{frame}[fragile]
  \frametitle{Euler–Maruyama method}
  Walk through the 3D free energy landscape with a stochastic walk
  (diffusion model like).
  \begin{lstlisting}
import random
def simulate(y_start, t_begin=0.0, t_end=0.002, dt=.0001,
            Bs=(8000, 8000, 8100)):
    #...
    for i in xrange(1,Nt):
        y[i] = (y[i-1] + dt*drift_rate(y[i-1], Bs) +
                sigma*sqrtdt*np.array((random.gauss(0,1),
                                    random.gauss(0,1),
                                    random.gauss(0,1))))
    #...
  \end{lstlisting}
  \begin{itemize}
      \item need \emph{drift rate}
      \item set \emph{all parameters} except Bs
  \end{itemize}
\end{frame}

\begin{frame}[fragile]
  \frametitle{Derivatives}
In order to get the drift rate we need the steepness of the free energy
landscape.
$$\frac{\partial F_{IDM}}{\partial y_1} =$$
  \begin{lstlisting}
>>> F_IDM.diff(y1)
-B1 + N1*(log(y1) - log(-y1 + 1))/beta + THETA1 + Wminus12*y2
+ Wminus13*y3 - 2*Wplus1*y1

>>> from sympy import latex
>>> latex(F_IDM.diff(y1))
  \end{lstlisting}
  $$- B_{1} + \frac{N_{1} (\operatorname{log}(y_{1}) -
\operatorname{log}(- y_{1} + 1))}{\beta} + \Theta_{1} +
W^-_{12} y_{2} + W^-_{13} y_{3} - 2 W^+_{1} y_{1}$$
Analogous for $\frac{\partial F_{IDM}}{\partial y_2}$ and $\frac{\partial
F_{IDM}}{\partial y_3}$
\end{frame}

\begin{frame}[fragile]
  \frametitle{Smart Guessing}
  Thanks to Steijn we randomly smart guessed for an evening and find a good
  parameter set.
  \begin{lstlisting}
param = {"Wplus1": 52000,
     "Wplus2": 52000,
     "Wplus3": 52000,
     "Wminus12": 12000,
     "Wminus13": 12000,
     "Wminus23": 12000,
     "THETA1": 52500,
     "THETA2": 52500,
     "THETA3": 52500,
     "N1": 1000,
     "N2": 1000,
     "N3": 1000,
     "beta": 1/24}
F_IDM.subs(param)
  \end{lstlisting}
\end{frame}

\begin{frame}[fragile]
  \frametitle{Smart Guessing}
  Thanks to Steijn we randomly smart guessed for an evening and find a good
  parameter set.
  \begin{lstlisting}
param = {"Wplus1": 52000,
     "Wplus2": 52000,
     "Wplus3": 52000,
     "Wminus12": 12000,
     "Wminus13": 12000,
     "Wminus23": 12000,
     "THETA1": 52500,
     "THETA2": 52500,
     "THETA3": 52500,
     "N1": 1000,
     "N2": 1000,
     "N3": 1000,
     "beta": 1/24}
F_IDM.subs(param)
  \end{lstlisting}
\end{frame}

\begin{frame}
  \frametitle{Something went wrong}
  \begin{itemize}
      \item Now we could run some simulations (but they were very slow and
          consumed much of the memory).
      \item Remember we wanted to mimic with systematically differing Bs
          human response behaviour to an SJ3 task.
      \item Trying to run a bulk of simulations during night failed due to
          memory consumption.
      \item But "Captain Flieg" and a short night later...
  \end{itemize}
\end{frame}

\begin{frame}
  \frametitle{Ballmer Peak}
  \begin{figure}[h]
      \includegraphics[width=100mm]{figs/ballmer_peak_mod.png}
      \caption{Ballmer Peak (modified). xkcd.com/323/ (cc)-by-nc}
  \end{figure}
\end{frame}

\begin{frame}
  \frametitle{Video2}
  Video :)
\end{frame}


\begin{frame}
  \frametitle{Goal}
  \begin{columns}
  \begin{column}{0.30\textwidth}
  \includegraphics[width=0.7\textwidth]{figs/checkbox-checked.png}
  \end{column}
  \begin{column}{0.70\textwidth}
  With an extended IDM model, simulate human behaviour in a three-alternative temporal judgment task
  \end{column}
  \end{columns}
  \begin{columns}
  \begin{column}{0.30\textwidth}
  \includegraphics[width=0.7\textwidth]{figs/checkbox-unchecked.png}
  \end{column}
  \begin{column}{0.70\textwidth}
  Fit Independent-Channels Model (Miguel \& Rocio) to our simulated data 
  \end{column}
  \end{columns}
\end{frame}

%%%%%%%%%%%%%%%%%%%%%%%%%%%%%%%%%%%%%%%%%%%%%%%%
%%% Tino end %%%%%%%%%%%%%%%%%%%%%%%%%%%%%%%%%%%
%%%%%%%%%%%%%%%%%%%%%%%%%%%%%%%%%%%%%%%%%%%%%%%%

\section{Integrated Independent-Channels Model (ICM)}
\label{sec:icm}

\begin{frame}
  \frametitle{Integrated Independent-Channels Model (ICM)}
  \begin{itemize}
   \item  Model distinction between unobservable judgments and
   observed responses; extension to different time judgment tasks (SJ2, SJ3, TOJ).  
   \item  Signals from 2 stimuli arrive at a central mechanism with 2
    exponentially distributed arrival latencies T1 and T2.
   \newline
  \begin{equation}
   g_i(t) = \lambda_i \text{exp}[-\lambda_i(t-(\Delta t_i + \tau_i ))],
   \quad t \leq \Delta t_i + \tau_i,\quad  i\in\{v,a\}
  \end{equation}
   \newline
   $\mu = \frac{1}{\lambda_i+\tau_i},\quad\sigma = \frac{1}{\lambda_i},
   \quad \Delta t_v=0, \quad\Delta t\equiv\Delta t_a (SOA) $
   \newline
   \item  Judgment of temporal order or synchrony is determined by
   a ternary decision rule applied to the arrival time
   difference $D=T_a-T_v$ between the 2 signals.
 \end{itemize}
\end{frame}

\subsection{Model parameters}

\begin{frame}
\frametitle{ICM - Model parameters (1)}
 \begin{columns}[t]
  \begin{column}{0.50\textwidth}
   \begin{figure}[h]
      \includegraphics[width=50mm]{figs/SJ3_Figure1.jpg}
      \caption{a) Arrival latency distribution at $T_i$; 
      b) Arrival-time differences distribution $D=T_a-T_v$}
    \end{figure}
  \end{column}
  \begin{column}{0.50\textwidth}
    \vfill
    Resolution parameter $\delta$ (in msec) describes the 
    observer's ability to resolve small differences in arrival 
    latency and defines two cutpoints in the decision space. 
    
    \begin{equation*}
    \begin{array}{l l l}
    D<-\delta & \Rightarrow & \text {AF judgment}\\
    -\delta\leq D\leq\delta & \Rightarrow & \text {S judgment}\\
    D < \delta & \Rightarrow & \text {VF judgment} 
    \end{array}
    \end{equation*}
  \end{column}
 \end{columns}
\end{frame}


\begin{frame}
 \frametitle{ICM - Model parameters (2)}
  Probability of each judgment as a function of auditory delay $\Delta t$ and resolution parameter $\delta$.
  \begin{equation}
   \begin{array}{l}
    \Psi_\text{SJ3 AF}(\Delta t)=\int_{-\delta}^{-\infty} f(z;\Delta t)dz=F(-\delta;\Delta t)\\
    \Psi_\text{SJ3 S}(\Delta t)=\int_{\delta}^{-\delta} f(z;\Delta t)dz=F(\delta;\Delta t)-F(-\delta;\Delta t)\\
    \Psi_\text{SJ3 VF}(\Delta t)=\int_{\infty}^{\delta} f(z;\Delta t)dz=1-F(\delta;\Delta t)
   \end{array}
  \end{equation}
  \begin{figure}[ht!]
   \includegraphics[width=65mm,totalheight=0.5\textheight]{figs/SJ3-psychometric-function.png}
  \end{figure}
\end{frame}


\begin{frame}
  \frametitle{ICM - Model parameters (3)}
   \begin{columns}[t]
    \begin{column}{0.50\textwidth}
    \begin{figure}[h]
      \includegraphics[width=50mm]{figs/SJ3_Figure3.jpg}
    \end{figure}
   \end{column}
  \begin{column}{0.50\textwidth}
   \vfill
   $\varepsilon_\text{AF,S,VF}$ accounts for the probability of 
   judgment misreports according to a response bias $K_\text{X-Y}$ which 
   describes the bias toward misreporting a response Y rather than Z $(1-K_\text{X-Z})$ to the judgment X.  
  \end{column}
 \end{columns}
\end{frame}


\subsection{ICM fit to real data}

\begin{frame}
  \frametitle{ICM - Fit to empirical data}
  Fitted ICM model to four subjects' real data: response proportion
  as a function of auditory delay (SOA)
   \begin{figure}[h]
      \begin{tabular}{c c}
      \includegraphics[width=45mm,totalheight=0.4\textheight]{figs/SJ3_Figure4a.jpg}
      &
      \includegraphics[width=45mm,totalheight=0.4\textheight]{figs/SJ3_Figure4b.jpg}
       \end{tabular}
   \end{figure}
\end{frame} 


\subsection{Fitting ICM to 3D IDM data}

\begin{frame}
  \frametitle{Fitting ICM to 3D IDM data}
  Fitting of ICM model to SJ3 data simulated via 3D IDM: 
  \begin{itemize}
   \item 3 different observers (2 normal and one particularly "skilled")
   \item 216 trials for each of 15 auditory delays ranging from -350 to 350 (steps of 50 msec)
   \item fit of the ICM full model for SJ3 task with error 
   parameters $\varepsilon_\text{AF,S,VF}$ and all different Ks 
  \end{itemize}
\end{frame}


\begin{frame}
  \frametitle{Fitting ICM to 3D IDM data: results (1)}
   \begin{figure}[h]
      \begin{tabular}{c c}
      Left & Right\\
      \includegraphics[width=45mm,totalheight=0.4\textheight]{figs/sub2_plot_Bs}
      &
      \includegraphics[width=45mm,totalheight=0.4\textheight]{figs/sub1_plot_Bs}\\
        
      \includegraphics[width=45mm,totalheight=0.4\textheight]{figs/sub2_fitted_model.jpg}
      &
      \includegraphics[width=45mm,totalheight=0.4\textheight]{figs/sub1_fitted_model.jpg}
       \end{tabular}
   \end{figure}
\end{frame}  

\begin{frame}
  \frametitle{Fitting ICM to 3D IDM data}
   \centering{A.W.E.S.O.M.O}
   \begin{figure}[h]
    \includegraphics[width=60mm,totalheight=0.65\textheight]{figs/funny-drunk-guy-robot.jpg}
    \end{figure}
\end{frame}


\begin{frame}
  \frametitle{Fitting ICM to 3D IDM data: results (2)}
   \begin{figure}[h]
      \begin{tabular}{c c}
      %bla bla  & bla bla\\
      \includegraphics[width=50mm,totalheight=0.5\textheight]{figs/sub3_plot_Bs}
      &
      \includegraphics[width=50mm,totalheight=0.5\textheight]{figs/sub3_fitted_model.jpg}
       \end{tabular}
   \end{figure}
\end{frame}


\begin{frame}
  \frametitle{Goal}
  \begin{columns}
  \begin{column}{0.30\textwidth}
  \includegraphics[width=0.7\textwidth]{figs/checkbox-checked.png}
  \end{column}
  \begin{column}{0.70\textwidth}
  With an extended IDM model, simulate human behaviour in a three-alternative temporal judgment task
  \end{column}
  \end{columns}
  \begin{columns}
  \begin{column}{0.30\textwidth}
  \includegraphics[width=0.7\textwidth]{figs/checkbox-checked.png}
  \end{column}
  \begin{column}{0.70\textwidth}
  Fit Independent-Channels Model (Miguel \& Rocio) to our simulated data 
  \end{column}
  \end{columns}
\end{frame}

\begin{frame}
  \frametitle{What have we seen?}
  \begin{itemize}
      \item How 2D Ising model can be extended to a 3D version, which can simulate behaviour in 3 alternative decision-making tasks.
      \item How we can easily model different behaviours by adjusting the strength of each attractor (B).
      \item How the IICM fit yields meaningful results.
  \end{itemize}
\end{frame}

\begin{frame}
  \frametitle{Speculation}
  \begin{itemize}
      \item Same mechanism for different tasks?
      \item Bump
      \item Reaction time
  \end{itemize}
\end{frame}

\begin{frame}
  \frametitle{ICM - Fit to empirical data}
  Fitted ICM model to four subjects' real data: response proportion
  as a function of auditory delay (SOA)
   \begin{figure}[h]
      \begin{tabular}{c c}
      \includegraphics[width=50mm]{figs/SJ3_Figure4a.jpg}
      &
      \includegraphics[width=50mm]{figs/SJ3_Figure4b.jpg}
       \end{tabular}
   \end{figure}
\end{frame}

\begin{frame}
  \frametitle{ICM - Fit to simulated data}
   \begin{figure}[h]
      \includegraphics[width=75mm]{figs/sub2_fitted_model.pdf}
   \end{figure}
\end{frame}


\begin{frame}
  \frametitle{Fitting ICM to 3D IDM data: results (3)}
  IDM simulated RT distribution for AF, S, and VF responses (\#Subject 2)
   \begin{figure}[h]
    \includegraphics[width=95mm]{figs/sub2_reaction_times2.pdf}
    \end{figure}
\end{frame}


\begin{frame}
  \frametitle{Discussion}
  \begin{figure}[h]
      \includegraphics[width=100mm]{figs/decision_space_3d.pdf}
  \end{figure}
\end{frame}

\begin{frame}
  \frametitle{Questions}
  \begin{figure}[h]
      \includegraphics[width=80mm]{figs/funny-drunk-guy-robot.jpg}
  \end{figure}
\end{frame}


%\begin{frame}
% \frametitle{Fitting IICM to 3D data: results 1}
%    \begin{equation}
%     f(d;\Delta t)=\big\{ 
%     \begin{array}{l l}
%      \small{\frac{\lambda_a\lambda_v}{{\lambda_a + \lambda_v}} \text{exp}[\lambda_v(d-\Delta t -\tau)]
%      & \quad \text{if} d\leq\Delta t +\tau \\
%      \small{\frac{\lambda_a\lambda_v}{{\lambda_a + \lambda_v}} \text{exp}}[-\lambda_a(d-\Delta t -\tau)] 
%      & \quad \text{if} d>\Delta t +\tau
%     \end{array}
%    \end{equation}
%\end{frame}    

\end{document}

